\documentclass{article}

\usepackage{fullpage}
\usepackage{amsmath, mathtools, mathpartir}

%%%%%%%%%%%%%%%%%%%%%%%%%%%%%%%%%%%%%%%%%%%%%%%%%%%%%%%%%%%%%%%%%%%%%%%%%%%%%%%%
% Syntax

% Sorts

\def\Exp{\text{Exp}}
\def\Typ{\text{Typ}}

% Expressions and Types

\newcommand\Lam[3]{\lambda #1\!:\!#2 . #3}
\newcommand\App[2]{#1~#2}
\def\Num{\normalfont\texttt{num}}

%%%%%%%%%%%%%%%%%%%%%%%%%%%%%%%%%%%%%%%%%%%%%%%%%%%%%%%%%%%%%%%%%%%%%%%%%%%%%%%%
% Semantics

% Rules

\def\TVar{\textsc{TVar}}
\def\TLam{\textsc{TLam}}
\def\TApp{\textsc{TApp}}
\def\TNum{\textsc{TNum}}
\def\TPlus{\textsc{TPlus}}
\def\VLam{\textsc{VLam}}
\def\VNum{\textsc{VNum}}
\def\SSAppCOne{\textsc{SSApp1}}
\def\SSAppCTwo{\textsc{SSApp2}}
\def\SSAppR{\textsc{SSLam}}
\def\SSPlusCOne{\textsc{SSPlusC1}}
\def\SSPlusCTwo{\textsc{SSPlusC2}}
\def\SSPlusR{\textsc{SSPlusR}}

% Judgments

\newcommand\hastype[3][\Gamma]{#1 \vdash #2 : #3}
\newcommand\isval[1]{#1~\mathsf{\normalfont val}}
\def\smallstep{\longrightarrow}

% Helpers

\newcommand\subst[3]{[#1/#2]#3}

%%%%%%%%%%%%%%%%%%%%%%%%%%%%%%%%%%%%%%%%%%%%%%%%%%%%%%%%%%%%%%%%%%%%%%%%%%%%%%%%
% Theorems and Proofs

\usepackage{amsthm}
\newtheorem{Theorem}{Theorem}
\newtheorem{Lemma}[Theorem]{Lemma}

\usepackage{pfsteps}
\renewcommand\byCasesEveryCase{}
\setlength{\proofrightwidth}{0.4\linewidth}

%%%%%%%%%%%%%%%%%%%%%%%%%%%%%%%%%%%%%%%%%%%%%%%%%%%%%%%%%%%%%%%%%%%%%%%%%%%%%%%%

\title{The Simply Typed Lambda Calculus}
\author{}
\date{}

\begin{document}

\maketitle

%%%%%%%%%%%%%%%%%%%%%%%%%%%%%%%%%%%%%%%%%%%%%%%%%%%%%%%%%%%%%%%%%%%%%%%%%%%%%%%%

\section{Formalism}

\subsection{Syntax}

\[
  \arraycolsep=0pt
  \begin{array}{l@{~~}c@{~~}l}
    \Exp &   e  & \Coloneqq x \mid \Lam{x}{\tau}{e} \mid \App{e}{e} \mid n \mid e + e \\
    \Typ & \tau & \Coloneqq \tau \to \tau \mid \Num \\
  \end{array}
\]

\subsection{Static Semantics}

$\boxed{\hastype{e}{\tau}}$

\begin{mathpar}
  \inferrule[\TVar]{
    x : \tau \in \Gamma
  }{
    \hastype{x}{\tau}
  }

  \inferrule[\TLam]{
    \hastype[\Gamma, x : \tau_1]{e}{\tau_2}
  }{
    \hastype{ (\Lam{x}{\tau_1}{e}) }{\tau_1 \to \tau_2}
  }

  \inferrule[\TApp]{
    \hastype{e_1}{\tau_{11} \to \tau_{12}}
    \\
    \hastype{e_2}{\tau_{11}}
  }{
    \hastype{ (\App{e_1}{e_2}) }{\tau_{12}}
  }

  \inferrule[\TNum]{
  }{
    \hastype[]{n}{\Num}
  }

  \inferrule[\TPlus]{
    \hastype{e_1}{\Num}
    \\
    \hastype{e_2}{\Num}
  }{
    \hastype{ (e_1 + e_2) }{\Num}
  }
\end{mathpar}

\subsection{Dynamic Semantics}

$\boxed{\isval{e}}$

\begin{mathpar}
  \inferrule[\VLam]{
  }{
    \isval{(\Lam{x}{\tau}{e})}
  }

  \inferrule[\VNum]{
  }{
    \isval{n}
  }
\end{mathpar}

\noindent
$\boxed{e \smallstep e}$

\begin{mathpar}
  \inferrule[\SSAppCOne]{
    e_1 \smallstep e_1'
  }{
    \App{e_1}{e_2} \smallstep \App{e_1'}{e_2}
  }

  \inferrule[\SSAppCTwo]{
    \isval{e_1}
    \\
    e_2 \smallstep e_2'
  }{
    \App{e_1}{e_2} \smallstep \App{e_1}{e_2'}
  }

  \inferrule[\SSAppR]{
    \isval{e_2}
  }{
    \App{ (\Lam{x}{\tau_{11}}{e_{12}}) }{e_2} \smallstep \subst{x}{e_2}{e_{12}}
  }

  \inferrule[\SSPlusCOne]{
    e_1 \smallstep e_1'
  }{
    e_1 + e_2 \smallstep e_1' + e_2
  }

  \inferrule[\SSPlusCTwo]{
    \isval{e_1}
    \\
    e_2 \smallstep e_2'
  }{
    e_1 + e_2 \smallstep e_1 + e_2'
  }

  \inferrule[\SSPlusR]{
    \isval{e_1}
    \\
    \isval{e_2}
  }{
    e_1 + e_2 \smallstep n
  }~{n = e_1 + e_2}
\end{mathpar}

%%%%%%%%%%%%%%%%%%%%%%%%%%%%%%%%%%%%%%%%%%%%%%%%%%%%%%%%%%%%%%%%%%%%%%%%%%%%%%%%

\newpage
\subsection{Properties and Proofs}

\resetpfcounter
\begin{Lemma}[Canonical Forms]
  \label{theorem:Canonical Forms}
  If \usepfcounter[CF:e is closed and well typed] $\hastype[]{e}{\tau}$
  and \usepfcounter[CF:e is a value] $\isval{e}$
  then all of the following are true:
  \begin{enumerate}
    \item
        If $\tau = \tau_{11} \to \tau_{12}$,
      then $e = \Lam{x}{\tau_{11}}{e_{12}}$
      and $\hastype[\Gamma, x : \tau_{11}]{e_{12}}{\tau_{12}}$.
    \item
        If $\tau = \Num$,
        then $e = n$.
  \end{enumerate}
\end{Lemma}

The Canonical Forms lemma says that
  if we know an expression is closed and well-typed,
  and it is a value,
  then we know which syntactic forms it could be.

\begin{proof}
  By rule induction on \pfref{CF:e is a value}:
  \begin{byCases}
    \case{\VLam}
      \begin{pfsteps}
        \item @{CF:VLam:step 1} \ldots
        \item @{CF:VLam:step 2} \ldots
        \item @{CF:VLam:step 3} \ldots    \BY{IH on \pfref{CF:VLam:step 1, CF:VLam:step 2}}
      \end{pfsteps}
    \case{\VNum}
  \end{byCases}
\end{proof}

%%%%%%%%%%%%%%%%%%%%%%%%%%%%%%%%%%%%%%%%%%%%%%%%%%%%%%%%%%%%%%%%%%%%%%%%%%%%%%%%

% TODO: The Inversion lemma (for typing judgments) says that if the conclusion of a (typing) rule is true, then so are its premises.

%%%%%%%%%%%%%%%%%%%%%%%%%%%%%%%%%%%%%%%%%%%%%%%%%%%%%%%%%%%%%%%%%%%%%%%%%%%%%%%%

% TODO: The (Closed) Substitution lemma says that substitution preserves types.

% % \resetpfcounter
% \begin{Lemma}[Substitution]
%   \label{thm:preservation}
%   If \usepfcounter[Substitution:x types e] $\hastype[x : \tau']{e}{\tau}$
%   and \usepfcounter[Substitution:e' types] $\hastype[]{e'}{\tau'}$
%   then $\hastype[]{\subst{e'}{x}{e}}{\tau}$.
% \end{Lemma}

%%%%%%%%%%%%%%%%%%%%%%%%%%%%%%%%%%%%%%%%%%%%%%%%%%%%%%%%%%%%%%%%%%%%%%%%%%%%%%%%

% \resetpfcounter
\begin{Theorem}[Progress]
  \label{theorem:Progress}
  If \usepfcounter[Progress:e is closed and well typed]~$\hastype[]{e}{\tau}$
  then either $e \smallstep e'$
  or $\isval{e'}$.
\end{Theorem}

The Progress theorem says a closed, well-typed expression is either a value or on its way to becoming one.

\begin{proof}
  By rule induction on \pfref{Progress:e is closed and well typed}:
  \begin{byCases}
    \item[\TVar]
  \end{byCases}
\end{proof}

%%%%%%%%%%%%%%%%%%%%%%%%%%%%%%%%%%%%%%%%%%%%%%%%%%%%%%%%%%%%%%%%%%%%%%%%%%%%%%%%

% TODO: The Preservation theorem says that stepping preserves types.

% \resetpfcounter
\begin{Theorem}[Preservation]
  % \label{thm:preservation}
  If \usepfcounter[Preservation:e types] $\hastype[]{e}{\tau}$
  and \usepfcounter[Preservation:e steps] $e \smallstep e'$
  then $\hastype[]{e'}{\tau}$.
\end{Theorem}

\end{document}